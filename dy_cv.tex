%% start of file `template.tex'.
%% Copyright 2006-2015 Xavier Danaux (xdanaux@gmail.com).
%
% This work may be distributed and/or modified under the
% conditions of the LaTeX Project Public License version 1.3c,
% available at http://www.latex-project.org/lppl/.


\documentclass[10pt,a4paper,sans,dvipsnames,roman]{moderncv}        % possible options include font size ('10pt', '11pt' and '12pt'), paper size ('a4paper', 'letterpaper', 'a5paper', 'legalpaper', 'executivepaper' and 'landscape') and font family ('sans' and 'roman')
%\moderncvstyle{casual}                             % style options are 'casual' (default), 'classic', 'banking', 'oldstyle' and 'fancy'
% \moderncvstyle{classic}
% \moderncvstyle{casual}
\moderncvstyle{banking}
\usepackage{xcolor}
\newcommand{\myGray}[1]{\textcolor{gray}{#1}}
% \usepackage[utf8]{inputenc}
% \usepackage[OT1]{fontenc}
% \usepackage{lmodern}
\setlength{\hintscolumnwidth}{0.15\textwidth}
\setlength{\separatorcolumnwidth}{0.025\textwidth}
% \setlength{\maincolumnwidth}{\textwidth-\leftskip-\rightskip-\separatorcolumnwidth-\hintscolumnwidth}%

% \setlength{\maincolumnwidth}{0.5\textwidth}%

\renewcommand*{\cvitem}[3][.25em]{%
  \begin{tabular}{@{}
    p{\hintscolumnwidth}@{\hspace{\separatorcolumnwidth}}p{\maincolumnwidth-\hintscolumnwidth-\separatorcolumnwidth}@{}}%
    \raggedleft\hintstyle{\raggedleft 
      \footnotesize\textcolor{gray}{#2}} &{ 
      #3}%
  \end{tabular}%
  \par\addvspace{#1}}

% \renewcommand*{\cvdoubleitem}[5][.25em]{%
%   \cvitem[#1]{#2}{%
%     \begin{minipage}[t]{\doubleitemcolumnwidth}#3\end{minipage}%
%     \hfill% fill of \separatorcolumnwidth
%     \begin{minipage}[t]{\hintscolumnwidth}\raggedleft\hintstyle{#4}\end{minipage}%
%     \hspace*{\separatorcolumnwidth}%
%     \begin{minipage}[t]{\doubleitemcolumnwidth}#5\end{minipage}}}

% \renewcommand*{\cvlistitem}[2][.25em]{%
%   \cvitem[#1]{}{\listitemsymbol\begin{minipage}[t]{\listitemcolumnwidth}#2\end{minipage}}}

% \renewcommand*{\cvlistdoubleitem}[3][.25em]{%
%   \cvitem[#1]{}{\listitemsymbol\begin{minipage}[t]{\listdoubleitemcolumnwidth}#2\end{minipage}%
%   \hfill% fill of \separatorcolumnwidth
%   \ifthenelse{\equal{#3}{}}%
%     {}%
%     {\listitemsymbol\begin{minipage}[t]{\listdoubleitemcolumnwidth}#3\end{minipage}}}}

\renewcommand*{\cventry}[7][.25em]{%
  \cvitem[#1]{#2}{%
    {\bfseries#3}%
    \ifthenelse{\equal{#4}{}}{}{, {\slshape#4}}%
    \ifthenelse{\equal{#5}{}}{}{, #5}%
    \ifthenelse{\equal{#6}{}}{}{, #6}%
    .\strut%
    \ifx&#7&%
    \else{\newline{}\begin{minipage}[t]{\linewidth}\small#7\end{minipage}}\fi}}


% \renewcommand*{\cvitemwithcomment}[4][.25em]{%
%   \savebox{\cvitemwithcommentbox}{{#3}}%
%   \setlength{\cvitemwithcommentskilllength}{\widthof{\usebox{\cvitemwithcommentbox}}}%
%   \setlength{\cvitemwithcommentcommentlength}{\maincolumnwidth-\separatorcolumnwidth-\cvitemwithcommentskilllength}%
%   \cvitem[#1]{#2}{%
%     \begin{minipage}[t]{\cvitemwithcommentskilllength}\usebox{\cvitemwithcommentbox}\end{minipage}%
%     \hfill% fill of \separatorcolumnwidth
%     \begin{minipage}[t]{\cvitemwithcommentcommentlength}\raggedleft\small\itshape#4\end{minipage}}}

\moderncvcolor{blue}                               % color options 'black', 'blue' (default), 'burgundy', 'green', 'grey', 'orange', 'purple' and 'red'
\usepackage[utf8]{inputenc}
\usepackage{lmodern,textcomp}
\usepackage[textwidth=6in, textheight=9.5in]{geometry}
%\usepackage[scale=0.75]{geometry}
\renewcommand{\refname}{Notable publications $\star$}



\newcommand\myshade{80}
\colorlet{mylinkcolor}{violet}
\colorlet{mycitecolor}{YellowOrange}
\colorlet{myurlcolor}{Aquamarine}


\usepackage{lastpage}
\rfoot{\addressfont\itshape\textcolor{gray}{\thepage\ / \pageref{LastPage}}}

\AfterPreamble{\hypersetup{
    pdftitle = {David Yallup CV},
  pdfsubject = {David Yallup CV},
  pdfkeywords = {CV},
  pdfauthor = {\textcopyright\ David Yallup},
  linkcolor  = black,%NavyBlue!\myshade!black, %!\myshade!black,
  citecolor  = Aquamarine!\myshade!black,
  urlcolor   = Aquamarine!\myshade!black,
  colorlinks = true
}}

% \@ifpackageloaded{hyperref}{%
% \hypersetup{%
%   pdftitle = {David Yallup CV},
%   pdfsubject = {David Yallup CV},
%   pdfkeywords = {CV},
%   pdfauthor = {\textcopyright\ David Yallup},
%   linkcolor  = black,%NavyBlue!\myshade!black, %!\myshade!black,
%   citecolor  = Aquamarine!\myshade!black,
%   urlcolor   = Aquamarine!\myshade!black,
%   colorlinks = true
% }


% personal data
\name{David}{Yallup}
%\title{CV}
\address{Kavli Institute for Cosmology, University of Cambridge }
\email{david.yallup@gmail.com}
\homepage{yallup.github.io}
\social[linkedin]{dyallup}                        % optional, remove / comment the line if not wanted
\social[github]{yallup}
% optional, remove / comment the line if not wanted
\extrainfo{\footnotesize{ORCiD: \href{https://orcid.org/0000-0003-4716-5817}{0000-0003-4716-5817}\quad Google Scholar: \href{https://scholar.google.com/citations?hl=en&user=gmQyi6EAAAAJ}{david.yallup}}}

%GitLab \href{https://gitlab.com/david.yallup}{david.yallup}}

% \moderncvicons{marvosym}
\makeatletter\renewcommand*{\bibliographyitemlabel}{\@biblabel{\arabic{enumiv}}}\makeatother

\begin{document}

\makecvtitle
% I am a postdoctoral researcher based in the cosmology group at the University of Cambridge. My work focuses on Bayesian Machine Learning and \emph{explainable AI}, with applications ranging from industrial challenges to fundamental science at both the biggest and smallest known scales in Physics.
{\begin{center} \textcolor{gray}{\footnotesize \textit{Research Themes: \textbf{Probabilistic Machine Learning, Explainable AI, Scientific applications of Machine Learning}}}\end{center}}
% \newline{}


%
\section{Appointments}
\cventry{2021-}{Postdoctoral Research Associate}{Kavli Institute for Cosmology}{University of Cambridge}{Cambridge}{
  \begin{itemize}
    \item Cosmology, High Energy Physics, accelerated likelihoods for Gravitational Wave Physics and CMB physics.
    \item Simulation Based inference, interfacing generative AI with scientific inference problems.
    \item MCMC methods, particularly with a view to building scalable GPU inference tools.
  \end{itemize}}
\cventry{2022-}{Corpus Christi College}{University of Cambridge}{Research Associate}{}{Associate member of college research community}
\cventry{2021-}{Polychord Ltd.}{}{Research associate}{}{Partnered with a Cambridge spin out startup data science consultancy, aiding development of novel Bayesian techniques for a wide array of industrial challenges.}
\subsection{Previously}
\cventry{2019-2020}{Postdoctoral Research Associate}{High Energy Physics group}{UCL}{London}{
Developing Machine Learning tools for inference over theoretical models at the highest energy frontiers. Assisting supervision of six masters students using software tools I wrote.
}
\cventry{2015-2019}{Doctoral candidate}{High Energy Physics group}{UCL}{London}{
  PhD student working on the ATLAS experiment and with the MCnet collaboration for collider physics theory. Leading development of tools for simulating collider physics, and analysing big data from the experiments for dark matter signals.
}
% \section{Associations}
% \subsection{Active}
% \cventry{2022-}{Corpus Christi College}{University of Cambridge}{Research Associate}{}{Associate member of college research community}
% \cventry{2021-}{Polychord Ltd.}{}{Research associate}{}{Partnered with a Cambridge spin out startup data science consultancy, aiding development of novel Bayesian techniques for a wide array of industrial challenges.}
% \cventry{2021}{Project Data Scientist}{Turing Institute}{London}{}{Selected participant in Data Study group, parterned with Odin Vision investigating explainable AI for cancer diagnosis.}
% \subsection{Previously}
\cventry{2015-2019}{MCnet Collaboration}{}{}{}{International multi-institue theory collaboration, core member of UCL node.}
\cventry{2015-2019}{ATLAS Experiment}{CERN}{}{}{Qualified author on the ATLAS experiment}
\cventry{2017}{Visiting researcher}{ATLAS Experiment}{}{}{STFC funding secondment at CERN for 9 months on the ATLAS experiment.}
\cventry{2016}{Visiting Researcher}{Karlsruhe Institue of Technology}{Institute of Theretical Physics}{}{Marie Curie early career visiting researcher four month project}
%\vspace{0.1cm}
\cventry{2013-2014}{Simcorp Ltd.}{Business Consultant}{Simcorp Ltd.}{London}{}

%\newpage

\section{Education}
\cventry{2015--2019}{PhD. Particle Physics}{UCL}{}{}{
  \textbf{Recipient of UCL HEP postgraduate prize for outstanding postgraduate research.} \\  
  Thesis titled, \textit{``Constraining new physics with fiducial LHC measurements.''} supervised by Prof J. Butterworth}  % arguments 3 to 6 can be left empty
\cventry{2014--2015}{MSc Particles, Strings and Cosmology}{Durham University}{}{}{}
\cventry{2009--2013}{MSci Natural Sciences, Maths and Physics}{Durham University}{}{}{}

\section{Grants}
\cventry{2024}{Kavli Focus Meeting}{$\sim\mathsterling2$k}{Kavli Institute for Cosmology Cambridge}{}{Funding to host Cosmological inference in High dimension workshop.}
\cventry{2017}{Marie Curie short term Early Stage Researcher grant}{$\sim\mathsterling30$k}{}{}{Fully funded Marie Curie visiting position at KIT for 5 months.}
\cventry{2015-2019}{MCnet mobility allowance}{Totalling $\sim\mathsterling5$k}{}{}{Awarded numerous travel grants for international conferences under MCnet Marie Curie network.}

% \section{Schools}
% \cventry{2018}{CERN-Fermilab Hadron Collider Physics summer school}{Fermilab}{USA}{}{}
% \cventry{2017}{MCnet summer school on Monte Carlo event generators for LHC physics}{Lund University}{Sweden}{}{}
% \cventry{2016}{STFC High Energy Physics summer school}{Lancaster University}{UK}{}{}
\section{Students}
\cventry{2023-24}{Pratyush Mishra}{University of Cambridge}{Part III Natural Sciences project}{}{Simulation Based inference bump hunting.}


\cventry{2022-23}{Namu Kroupa}{University of Cambridge}{Part III Natural Sciences project}{}{Marginalised Gaussian Processes for Cosmology}

\cventry{2022-23}{Boris Deletic}{University of Cambridge}{Part III Natural Sciences project}{co-supervised w. Dr. Will Barker}{Modified gravity on the lattice}

\cventry{2018}{Harry Saunders}{UCL}{MSc in Scientific Computing}{Lead superviser Profesor Jon Butterworth}{Adaptive sampling for physics models in high dimension}

\section{Teaching}

\subsection{Undergraduate teaching}
\cventry{2025}{Part II Statistical Physics}{University of Cambridge}{Institute of Astronomy (Maths)}{}{4 groups of 2 students, $\sim$ 40 hours}
\cventry{2022}{Part II Relativity}{University of Cambridge}{Natural sciences tripos (Physics)}{}{4 groups of 3 students, $\sim$ 40 hours}
\cventry{2015-2019}{Masters project supervision}{UCL}{MSci Physics projects}{}{Lead supervisor Prof. Jon Butterworth, assisted over 10 project students}
\cventry{2016}{First year physics lab demonstrator}{UCL}{}{}{}

\subsection{Technical teaching}
\cvitem{2019}{Original author of software tutorials for \textsc{Rivet} and \textsc{Contur} particle physics packages. Delivered at two MCnet summer schools.}
\cvitem{2019}{Young Experiment and Theorist Institute school tutor on \textsc{Rivet} and \textsc{Contur}.}
\cvitem{2018-2019}{ATLAS UK meeting tutor on the \textsc{Rivet} package and Monte Carlo methods for particle physics.}

\section{Dissemination}
\subsection{Invited Talks}
\cventry{2025}{ICLR: Frontiers of Probabilistic Inference}{Singapor}{}{}{Diffusion Meets Nested Sampling}
\cventry{2024}{BayesAI Workshop}{Lancaster University}{}{}{Diffusion Meets Nested Sampling}
\cventry{2024}{EUCAIFCon}{Amsterdam}{}{}{Diffusion Meets Nested Sampling}
\cventry{2024}{Hills Coffee Talks}{University of Cambridge}{}{}{Diffusion Meets Nested Sampling}
\cventry{2024}{Cambridge Astrophysics Machine Learning Seminar}{University of Cambridge}{}{}{Simulation Based Inference}
\cventry{2023}{Voyages Beyond the Standard Model}{Athens}{}{}{}
\cventry{2023}{UKLFT}{University of Cambridge}{}{}{Nested Sampling for Lattice Field Theory}
\cventry{2022}{Bayesian Inference in High Energy Physics}{Durham University}{}{}{}
\cventry{2022}{Likelihood Free in Paris}{L'\'Ecole Normale Supérieure}{Paris}{France}{}
\cventry{2021}{Learn the Universe - LFI for Cosmology}{Flatiron Institute}{NY}{USA}{}
\cventry{2019}{Les Houches Physics at TeV Colliders}{Chamonix}{France}{}{}%Invited attendee}
\cventry{2019}{ATLAS Exotics workshop}{Naples}{Italy}{}{}%Constraints on new theories from precision measurements}
\cventry{2019}{Rencontres de Moriond - EW Interactions and Unified Theories}{La Thuile}{Italy}{}{}%BSM Constraints from Standard Model measurements with Contur}
\cventry{2019}{Young Experimentalist and Theorists Institute}{Durham}{UK}{}{}%BSM Constraints from Standard Model measurements with Contur}
\cventry{2018}{Institute of Physics annual meeting}{Bristol}{UK}{}{}%Reinterpretation of precision LHC measurements}
\cventry{2018}{MC4BSM}{Durham}{UK}{}{}%{MC simulation for BSM physics in ATLAS.}
\cventry{2017}{Alpine LHC Summit}{Innsbruck}{Austria}{}{}%Constraints on New Theories Using Rivet}

\subsection{Invited Seminars}
\cventry{2023}{Evidence is all you need: Nested Sampling for particle physics}{University College London}{High Energy Physics group seminar}{}{}
\cventry{2022}{Evidence is all you need: Nested Sampling for particle physics}{University of Cambridge}{High Energy Physics seminar}{}{}

\subsection{Outreach}
\cventry{2018}{Science Centre Lectures}{University College London}{Physics at the energy frontier}{}{Outreach talk to a group of over 100 sixth form students}
\cventry{2017-2019}{Life at CERN}{University College London}{}{}{Research introduction talks to third year students in physics at UCL.}

%\section{Publications}
\nocite{*}
%\bibliographystyle{unsrtdin}
%\bibliographystyle{h-physrev5}
\bibliographystyle{JHEP}
\bibliography{dy_cv} 
\vspace{1cm}
\cvitem{}{\small{\emph{$\star$ As an ATLAS collaboration author I was an author on over 280 collaboration papers, only external small authorlist papers are listed here. Inclusion here represents a first author level contribution.}}}
\end{document}



